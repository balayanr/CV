% !TeX program = xelatex
% Run with XeLaTeX
% You can change the base color of this cv by altering the RGB values of \definecolor{maincolor}{RGB}{102, 204, 51} in class file:  cv-roald.cls. It will automatically define a darker and lighter shade of this color. The darker shade is used for the background of the header, the lighter for the contact details in the header and the maincolor is used for the titles. 

\documentclass[]{cv-roald}
% \usepackage[utf8]{inputenc}
% \usepackage[english, japanese]{babel}
% \usepackage[C..]{fontenc}
\usepackage{xeCJK}
\setCJKmainfont{AppliMincho}
\usepackage{hyperref}
\hypersetup{
    colorlinks=true,
    urlcolor=\color{maincolor}
}

% Toggles:
% Columbia - only for gaming-related companies
% Natural languages in summary
% SMT in summary
% Toggles for all languages and experience qualifiers
% CFG and Formal Languages in skills
% Knowledge of NLP in skills
% Praat in skills
% Latex in skills
% Videogames in interests 


\begin{document}

\pagestyle{empty} %to remove the page numbers

% This is the header of the first page, which contains your name and contact details. 
% \sep inserts a | between items. 
% You can use FontAwesome icons and use \FAspace after a font awesome icon to insert a predefined horizontal space after a font awesome icon icon.
\header{Robert}{Balayan}
{\faMapMarker \hspace{\FAspace} Japan, Ōsaka-fu, Ōsaka-shi, Jotō-ku Furuichi 1-21-22 \#102}
{\faMobile \hspace{\FAspace} +81 80 7802 6118 \sep 
\href{mailto:robert@balayanr.com}{\faEnvelope} \hspace{\FAspace} \faSkype\hspace{\FAspace} robert@balayanr.com \sep 
\href{https://www.linkedin.com/in/balayanr/}{\faLinkedinSquare} \hspace{\FAspace} \href{https://github.com/balayanr}{\faGithub} \hspace{\FAspace} \href{https://www.facebook.com/balayanr}{\faFacebookSquare} \hspace{\FAspace}\href{https://vk.com/yahhh_gf}{\faVk} \hspace{\FAspace} balayanr \FAspace  %
}
%\href{https://vk.com/yahhh_gf}{\faVk}\hspace{\FAspace} \href{twitter.com/yahhh_gf}{\faTwitterSquare}\hspace{\FAspace} \href{instragram.com/yahhh_gf}{\faInstragram} yahhh_gf
\hypersetup{
    urlcolor=darkermain
}


\textit{AI and Software Engineer in Osaka, Japan. Looking to work with Data and Machine Learning. Interested in Database Architecture and Distributed Systems.}

\section*{summary}
\begin{tabitemize}
    \item Experienced in Machine Learning, Natural Language Processing and Computational Linguistics, General AI Theory, and Knowledge Representation.
    \item Programming Languages: Python, SQL, PL/pgSQL, C, Java, Matlab, R.
    \item Natural Languages: English, Russian, Armenian (Conversational), Japanese (Basic).
    \item Frameworks and tools: NumPy, TensorFlow, keras, AWS (RDS, EC2, DyDB, S3), Luigi, Selenium, BeautifulSoup, Kubernetes, Docker, Hadoop, NLTK, Django REST, \LaTeX, Draw.io for ERD design.
    % \item Solid mathematical and statistical background.
\end{tabitemize}


\section*{work experience}
% Use tabularcv environment to make a two column environment, left one for dates, right one for details of your education for example. 
% You can use the command \worktitle{Study name/Job title}{Location}.
% You can use the environment tabitemize to make a bulletpoint list inside the tabularcv environment.
\begin{tabularcv}
   April 2018 - February 2020       &   \worktitle{Lead AI Engineer}{Kuon Technologies, Osaka, Japan (クオンテクノロジーズ株式会社)}
                    \newline Lead engineer of the AI of the company's main product - Beep Shift, a scheduling AI that assigns part-time employees to the positions required by different businesses. The product is in its final stages of development with further expansion planned to include sales and requirement prediction.
                    \newline Handled everything related to the AI side of the project: from database architecture and API design to AI development and deployment.
                    \\[\vspacepar]
%   November 2019 - Now   & \worktitle{Freelance Software Engineer}{Freelance, Osaka, Japan}
%                     \newline I started taking occasional projects that help me learn new tools while fulfilling them. Through them I got proficient at Data Mining and Web scraping, further improved my understanding of Databases, ETL pipelines. Projects themselves are listed further below
%                     \\[\vspacepar]
  February 2020 - Now   & \worktitle{Lead Software Engineer}{K.K. Tree, Osaka, Japan  (株式会社TREE)}
                    \newline Lead the team handling web-scraping. Designed the whole ETL pipeline around the \href{https://luigi.readthedocs.io/en/stable/}{Luigi} framework, which allowed us to maximize server utilization with great scalability as most tasks can run in parallel.
                    \newline Designed and implemented a Python framework that allowed for new crawlers to be created in minutes with less than 20 LOC while still providing great flexibility for complex websites.
                    \newline Designed the Database used for the project in PostgreSQL, with S3 for file storage.
\end{tabularcv}

Projects are explained in greater detail in a section below.

\section*{education}
% Use tabularcv environment to make a two column environment, left one for dates, right one for details of your education for example. 
% You can use the command \worktitle{Study name/Job title}{Location}.
% You can use the environment tabitemize to make a bulletpoint list inside the tabularcv environment.
\begin{tabularcv}
    2014-2017   &   \worktitle{Honours Bachelor of Science}{University of Toronto, Canada}
                    \newline Computer Science Specialist with Focus in Artificial Intelligence
                    \newline Fields studied:
                    \begin{tabitemize}
                        \item Machine Learning, Convolutional and Recurrent Neural Networks
                        \item Computational Linguistics and Formal languages, Natural Language Processing, Phonetics
                        \item General AI theory, Knowledge Representation and Reasoning 
                        \item Graph Theory and Algorithm Design
                        \item Statistics, Linear Algebra and Multivariable Calculus\newline
                    \end{tabitemize}
                    Relevant Projects:
                    \begin{tabitemize}
                        \item CSC412 Solo Research Project: Exploiting Structure For Classification Of Handwritten Japanese Characters
                        \item CSC384 Group Research Project: CSP Algorithm Analysis
                        \item CSC343 Solo Project: SQL-based course recommendation engine
                    \end{tabitemize}
                    \\[\vspacepar] % Start new row with this
                    
%    2012-2014   &   \worktitle{Accelerated High School Completion \& University Transfer Program}{Columbia College, Vancouver, Canada}
%                    \newline Completed grades 11 and 12 in two semesters and then studied various first year science courses to later transfer to University of Toronto
%                    \begin{tabitemize}
%                        \item Studied basic Computer Science, Calculus and Introduction to Physics
%                        \item Created Video Gaming Club and hosted weekly meetings for 2 years
%                    \end{tabitemize} 
%                    \\[\vspacepar]
    2017-       &   \worktitle{Certifications}{Coursera}
                    \newline Hadoop Platform and Application Framework, \href{https://www.coursera.org/account/accomplishments/verify/D6W6PEBPALWW}{License D6W6PEBPALWW}
\end{tabularcv}

\clearpage

\section*{skills and experiences}
\begin{tabitemize}
    \item 6+ years of experience \textbf{programming in Python}: from manipulating SQL databases to Neural Networks and Deep Learning using Numpy, TensorFlow and Keras to Chart parsing and word sense disambiguation using NLTK to Data Mining with Selenium and BeautifulSoup. This is my language of choice for both personal projects and research. 
    % \item 
    % Additional experiences with \textbf{Machine Learning in Matlab} 
    % (statistical machine translation, acoustic perception, Hidden Markov Models, Mixture of Gaussians); 
    % limited experience with \textbf{low-level programming in C/C++} 
    % (Kernel extensions, file-system manipulation, computer graphics in OpenGL); 
    % limited experience with \textbf{Object-Oriented Programming in Java} 
    % (Android Application back-end development following the MVC paradigm, JDBM for PostgreSQL); 
%    basic \textbf{data analysis in R}.
    % \item Experience designing both discriminative and generative models for image classification, voice recognition and language modeling. Deep understanding of the Machine Learning theory and algorithms, as well as of limitations imposed by hardware. 
    % \item Experience with machine translation. Implemented IBM Model-1 and trained it on Hansard corpus for English-French translation.
    \item Interest in Natural Language Processing and Computational linguistics. Familiarity with basics of Information Retrieval, Speech Recognition and Synthesis, Word sense disambiguation, statistical chart parsing disambiguation, and Question Answering.
    \item Experience designing relational databases and implementing them in PostgreSQL, MySQL and SQLite. Experience with AWS storage solutions like DynamoDB and S3. Experience with \textbf{HDFS and HBase} as part of Hadoop Platform and Application Framework certification.
%    \item Experience building Context Free Grammars and Automatas for abstractly defined languages as well as approximating natural languages. Experience extending natural language CFGs with features and using these grammars for parsing with \href{http://www.cs.toronto.edu/~gpenn/ale.html}{The Attribute Logic Engine}. 
    %\item Experience analyzing speech recordings using Praat. 
    %\item Experience formatting publications and documents using \LaTeX.

\end{tabitemize}

                    
    

\section*{projects}
\begin{tabularcv}
    February 2020 - Now &       \textbf{Web scraping} @ K.K. Tree\newline 
                                Designed and implemented an ETL pipeline for a website that unifies listings from government tender websites.
                                The project is designed around \href{https://luigi.readthedocs.io/en/stable/}{\textbf{Luigi}}, which schedules tasks to maximize server utilization. Each task handled a different website: loaded and navigated the page with \textbf{Selenium}, parsed the data by locating elements with XPATHs and used \textbf{Tabula} and \textbf{PDFMiner} to extract data from PDFs. The data is then handed back to Luigi in CSV files to be inserted into the \textbf{PostgreSQL database} in batches to optimize load. \textbf{Psycopg2} is used to get data from the database, \textbf{Boto3} is used to talk to AWS S3 to store PDFs and screenshots.
                                % \newline Frameworks and libraries used for the project: Luigi, Selenium, BeautifulSoup, Tabula+Pandas, PDFMiner; Psycopg2 and Boto3. 
                                \newline Deployed on AWS EC2; Database is in PostgreSQL, deployed on AWS RDS; AWS S3 is used to store files for listings.
    \\[\vspacepar]
    % January 2020 &              \textbf{Database development for a system used to manage staff, tasks and billing.}\newline 
                                % Written in PostgreSQL, deployed on AWS RDS, as the rest of the application was using AWS Elastic Beanstalk.
    \\[\vspacepar]

    November 2019 &             \textbf{Crawling of Hello Work's Job Listings.}\newline 
                                A customer who wanted \href{https://www.oshigoto-shokai.jp/}{a page} that only contained listings relevant to their services. \newline 
                                Everything was deployed on AWS Lambda, with functions handling both server-side rendering for pages and crawling. Used BeautifulSoup for page parsing and Selenium to help with navigation, database in PostgreSQL and deployed on AWS.
    \\[\vspacepar]
    
    June 2018 - January 2020 & \textbf{Beep Shift} @ Kuon Technologies\newline 
                                At Kuon Technologies I worked on creating an AI that would assign employees to shifts created by the manager. The shifts were represented as a step curve, with number of people required at each time. The AI would transform that into a Constraint Satisfaction Problem, with time constraints on employees (Overtime, maximum salary, etc) and on the shop (budget, position requirements). The model also allowed for managerial preferences, which the AI interpreted as a selection order via a heuristic function.
                                \newline Once the AI was ready, our customers provided us with text files with raw receipt printout data, which I had to parse and load into a database, perform feature selection and extraction from the parsed data and create a model that would predict the number of employees required to run the shop.
    \\[\vspacepar]
    CSC412 & \textbf{Solo Research Project: Exploiting Structure For Classification Of Handwritten Japanese Characters.}
    \begin{tabitemize}
        \item Developed a parser for the \href{http://etlcdb.db.aist.go.jp/}{Electrotechnical Laboratory} datasets, which come incredibly compressed and required a bespoke decoding library.
        \item Tested 4 different architectures with 2 different goals: direct kanji classification and multilabel classification of radicals that make up the kanji.
        \item Used AWS EC2 p2.xlarge and c4.8xlarge instances to train the models in parallel.
        \item Compiled the report in \LaTeX, generally following standards used for publications.
        \item Published all source code and the report on \href{https://github.com/balayanr/kanji_recognition_412}{Github}
    \end{tabitemize}\\[\vspacepar]
    CSC384 & \textbf{Group Research Project: Constraint Satisfaction Problem Algorithm Analysis.}
    % \begin{tabitemize}
        \newline Was the \textit{de facto} team leader: provided the team with a selection of algorithms outside of the course's scope that would be interesting to analyze. Final selection was iterative improvement, path consistency, and a tree-specific algorithms.
        % \item Compiled the report in \LaTeX.
    % \end{tabitemize}
    % \\[\vspacepar]
    % CSC343 & Solo Project: SQL-based course recommendation engine. \href{https://github.com/balayanr/course_recommender_343}{(Github)}\\[\vspacepar]
    % CSC321 & Solo Project: Face classification using AlexNet, implemented in TensorFlow. \\[\vspacepar]
\end{tabularcv}

\section*{languages}
\begin{tabularcv}
    English			&	Fluent \\[\vspacepar]
    Russian			&	Fluent \\[\vspacepar]
    Armenian		&	Conversational \\[\vspacepar]
    Japanese        &   Basic
\end{tabularcv}

\section*{interests and hobbies}
\textbf{Artificial Intelligence}: I have always been interested with AI and inspired by its portrayal in science fiction. It is my dream to work on implementing a general AI. One of the most interesting roadblocks to me is the way humans interact with AI: in science fiction it is almost always seamless, like talking to another person. Personally I have always had interest in Natural Languages, so this is one of the fields I would like to study in depth.\\
\textbf{Sports}: I am an avid cyclist, runner and hiker. I used to train with the University of Toronto Road Racing team and participate in various local cycling events. Hiked to the top of Mount Fuji, as well as many mountains in Kansai region. I also enjoy diving and have various advanced diving certifications. \\
\textbf{Traveling}: I enjoy traveling whenever I have the opportunity. I have visited many countries, including Japan, Spain, France, Italy, England, Germany, Egypt, Turkey, Georgia and Armenia. Lately, I have been traveling with my road bike and touring in the places I visit.\\
\textbf{GunPla}: One of my favourite hobbies is building plastic model kits. Over time I have built dozens of kits from Mobile Suit Gundam and Warhammer 40K figurines.\\
% \textbf{Magic the Gathering}: I enjoy the card game as a challenge for building the best deck possible out of limited resources provided. Occasionally participate in tournaments.\\
%\textbf{Videogames}: I have interest in videogames of all genres, but my favourite games are mostly strategy-focused (Civilization, FTL: Faster Than Light, StarCraft 2). 
\end{document}